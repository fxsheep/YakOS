%!TEX TS-program = xetex
%!TEX encoding = UTF-8 Unicode

% Text from http://www.cnd.org/Classics/Novels/San_Guo/
% converted to Unicode and minimally marked-up for TeX by Jonathan Kew

\XeTeXlinebreaklocale "zh"
\XeTeXlinebreakskip = 0pt plus 1pt minus 0.1pt
\parindent = 24pt
\baselineskip = 14pt
\parskip = 0pt plus 2pt
\widowpenalty = 9000
\clubpenalty = 9000

\newif\ifVertical \Verticalfalse

\def\papersize{6in,9in}

\ifVertical % set parameters for vertical layout

  \hsize=7in
  \vsize=4.5in
  \hoffset=-.5in

  % macro to rotate a box of Chinese text set with the "vertical" font attribute
  \def\ChineseBox#1{\setbox0=\vbox{\boxmaxdepth=0pt
       #1}\dimen0=\wd0 \dimen2=\ht0
    \vbox to \dimen0{\hbox to \dimen2{\hfil
      \special{x:gsave}\special{x:rotate -90}\rlap{\vbox to 0pt
        {\box0\vss}}\special{x:grestore}}\vss}}

  \def\ChineseOutput{\shipout \vbox{\ifnum\pageno=1 \special{papersize=\papersize}\fi
      \ChineseBox{\makeheadline \pagebody \makefootline }}
      \advancepageno \ifnum \outputpenalty >-20000 \else \dosupereject \fi}

  \output={\ChineseOutput}

  \def\V{:vertical} % for font defs

  % use punctuation chars from BiauKai font, because positioning seems better for vertical text
  \font\punc="BiauKai/AAT\V" at 12pt
  \def\P#1{\catcode`#1=\active \uccode`~=`#1 \uppercase{\def~{{\punc\char`#1}\penalty0 }}}
  \P{,}\P{:}\P{;}\P{。}\P{、}\P{?}\P{!}

\else % horizontal layout
  
  \special{papersize=\papersize}

  \vsize=7in
  \hsize=4.5in
  \hoffset=-.25in
  \def\V{}

\fi

\font\body="STKaiti/AAT\V" at 12pt \body
\font\bold="STHeiti/AAT\V" at 12pt
\font\title="STHeiti/AAT\V" at 18pt
\font\small="STKaiti/AAT\V" at 10pt

\footline={\hfil{\small\folio}\hfil}

\catcode13=\catcode`\% % ignore newlines (don't want space there)

\centerline{\title YakOS实时操作系统的实现}
\bigskip
\catcode`\ =\catcode32 % treat ideographic space as a normal space

\centerline{\bold 李燕清}
\centerline{\bold yannik520@gmail.com}
\medskip
\leftline{诗曰:}

     古人学问无遗力,少壮工夫老始成。
\par
  纸上得来终觉浅,绝知此事要躬行。
\par
\medskip
\centerline{\bold 第一章}
\smallskip
\centerline{\bold 最简操作系统}
\medskip

  记得大学上学时学习C语言是从一个HelloWorld程序开始,其实不管是学习C语言还是学习其他,都遵循由简单到复杂的原则。今天我们开始实现一个操作系统也一样,首先我们的目标是实现一个最简操作系统,那何谓最简操作系统,我认为最简操作系统就是一个单进程的无限循环。不要小瞧这个无限循环,在51单片机盛行的那个年代,N多的系统就是由一个无限循环实现的,随着时代的发展,单一的无限循环再不能满足需求,操作系统也就随着人们的需求变的越来越复杂,操作系统的定义也就越来越清晰,操作系统担负起了整个系统的资源调度、管理的职责。
\par
  
	YakOS最终的目标不是实现一个简单的无限循环,但目前YakOS的迈出的第一步就是实现一个无限循环,更准确的说是在ARM系统上实现一个最简的操作系统。为什么要在选择ARM,理由很简单,ARM是使用最广的MCU,ARM就是主流,当然咱也不是随风逐流的人,你懂得!
\par
	
  在实现最我们的最简操作系统之前,让我们来先了解一下ARM微处理器。ARM系列处理器包括ARM7,ARM9,ARM11,A8/9 等,我们重点介绍ARM9,ARM9支持32位的ARM指令集及16位的Thumb指令集,在了解指令集之前,我们先了解一下ARM的寄存器。
\par

  。
\par

  。
\par

  。
\par

  。
\bye
